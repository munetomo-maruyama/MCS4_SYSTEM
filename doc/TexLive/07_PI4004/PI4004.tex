%==============================================================
In this chapter, we develop and execute a program in 4004 CPU assembly to compute 500 digits of the mathematical constant $\pi$. Despite being the world's first microprocessor developed during the dawn of microcontroller technology, the Intel 4004 exhibits surprisingly sufficient computational capability—even with its primitive instruction set.

Through this experiment, we aim to:
\begin{itemize}
  \item Showcase the computational potential of the 4004 CPU despite its limitations.
  \item Implement high-precision arithmetic routines within the constraints of a 4-bit architecture.
  \item Demonstrate practical applications of early microprocessors for complex numerical tasks.
  \item Celebrate the architectural elegance and historical significance of the MCS-4 system.
\end{itemize}

The challenge underscores that even vintage systems can tackle sophisticated calculations with careful programming, optimization, and ingenuity.

%==============================================================
\section{Principle of Pi Calculation}

This program utilizes Machin's formula to calculate the digits of $\pi$. Although the formula converges slowly, it is well-suited for implementation on simple microcontrollers.

Machin's formula is expressed as follows:


\begin{equation}
\frac{\pi}{4} = 4\arctan\left(\frac{1}{5}\right) - \arctan\left(\frac{1}{239}\right)
\end{equation}


That is,


\begin{equation}
\pi = 16\arctan\left(\frac{1}{5}\right) - 4\arctan\left(\frac{1}{239}\right)
\end{equation}



The arctangent function can be expanded into the infinite series:


\begin{equation}
\arctan\left(\frac{1}{p}\right) = \frac{1}{p} - \frac{1}{3p^3} + \frac{1}{5p^5} - \frac{1}{7p^7} + \cdots
\end{equation}



Therefore, $\pi$ can be computed as:

\begin{multline}
\pi = \left(16 \cdot \frac{1}{5} - 4 \cdot \frac{1}{239}\right)
    - \left(16 \cdot \frac{1}{3 \cdot 5^3} - 4 \cdot \frac{1}{3 \cdot 239^3}\right)\\
    + \left(16 \cdot \frac{1}{5 \cdot 5^5} - 4 \cdot \frac{1}{5 \cdot 239^5}\right)
    - \left(16 \cdot \frac{1}{7 \cdot 5^7} - 4 \cdot \frac{1}{7 \cdot 239^7}\right)
    + \cdots
\end{multline}



Which can also be written in the following form:


\begin{multline}
\pi = \frac{1}{1}\left(\frac{16 \cdot 5}{(5^2)^1} - \frac{4 \cdot 239}{(239^2)^1}\right)
    - \frac{1}{3}\left(\frac{16 \cdot 5}{(5^2)^2} - \frac{4 \cdot 239}{(239^2)^2}\right)\\
    + \frac{1}{5}\left(\frac{16 \cdot 5}{(5^2)^3} - \frac{4 \cdot 239}{(239^2)^3}\right)
    - \frac{1}{7}\left(\frac{16 \cdot 5}{(5^2)^4} - \frac{4 \cdot 239}{(239^2)^4}\right)
    + \cdots
\end{multline}

If multi-byte addition, subtraction, and division operations can be implemented, it becomes feasible to compute $\pi$ to a large number of digits.

%==============================================================
\section{Pi Calculation Algorithm}
This algorithm computes $\pi$ digit-by-digit using decimal arrays. Each array element stores a single decimal digit.

For example:
- Indexes \texttt{[0--3]} store the integer part.
- Indexes \texttt{[4+]} store the fractional part.

\vspace{1em}
\noindent \textbf{Arrays used:}
\begin{itemize}
  \item \texttt{PI[]}: Stores the value converging to $\pi$.
  \item \texttt{T1[]}: Computed as $(16 \times 5)/(5 \times 5)^n$
  \item \texttt{T2[]}: Computed as $(4 \times 239)/(239 \times 239)^n$
  \item \texttt{T3[]}: Computed as $(-1)^{n+1} \cdot \frac{1}{2n - 1} \cdot (\texttt{T1} - \texttt{T2})$
\end{itemize}

\vspace{1em}
\noindent \textbf{Computation Procedure:}
\begin{enumerate}
  \item $n = 0$: Compute \texttt{T1[]} and \texttt{T2[]}. Derive \texttt{T3[]} from \texttt{T1[]} and \texttt{T2[]}. Initialize \texttt{PI[]} to zero.
  \item $n = 1$: Compute \texttt{T1[]} and \texttt{T2[]}. Compute \texttt{T3[]} and add to \texttt{PI[]}.
  \item $n = 2$: Compute \texttt{T1[]} and \texttt{T2[]}. Compute \texttt{T3[]} and subtract from \texttt{PI[]}.
  \item $n = 3$: Compute \texttt{T1[]} and \texttt{T2[]}. Compute \texttt{T3[]} and add to \texttt{PI[]}.
  \item $n = 4$: Compute \texttt{T1[]} and \texttt{T2[]}. Compute \texttt{T3[]}. If \texttt{T3[]} is zero, convergence is achieved.
\end{enumerate}

\vspace{1em}
\noindent \textbf{Example Calculation (Up to Four Decimal Places):}

\begin{table}[h!]
\centering
\begin{tabular}{|c|c|c|c|c|}
\hline
$n$ & \texttt{T1[]} & \texttt{T2[]} & \texttt{T3[]} & \texttt{PI[]} \\
\hline
0 & 0080.0000 & 0956.0000 & 0000.0000 & 0000.0000 \\
1 & 0003.2000 & 0000.0167 & 0003.1833 (+) & 0003.1833 \\
2 & 0000.1280 & 0000.0000 & 0000.0426 (-- ) & 0003.1407 \\
3 & 0000.0051 & 0000.0000 & 0000.0010 (+) & 0003.1417 \\
4 & 0000.0002 & 0000.0000 & 0000.0000 (-- ) & 0003.1417 \\
\hline
\end{tabular}
\caption{Values of \texttt{T1[]}, \texttt{T2[]}, \texttt{T3[]}, and \texttt{PI[]} for Each $n$}
\end{table}

%==============================================================
\section{RAM Data Assignment}
We assign memory data to RAM in the MCS-4 system for the computation of $\pi$.

The array RAMCH[Bank][SRC] stores four primary arrays: \texttt{T1[]}, \texttt{T2[]}, \texttt{T3[]}, and \texttt{PI[]}. The available RAM size is:

\[
8~\text{banks} \times 4~\text{chips} \times 4~\text{registers} \times 16~\text{characters} = 2048~\text{characters}
\]

Since there are four arrays, each is allocated 512 elements. This defines the precision to 500 digits of $\pi$. The arrays are assigned to RAM banks as follows:

\begin{itemize}
  \item \texttt{PI[512]}: Bank 0 (SRC=0x00--0xFF), Bank 1 (SRC=0x00--0xFF)
  \item \texttt{T1[512]}: Bank 2 (SRC=0x00--0xFF), Bank 3 (SRC=0x00--0xFF)
  \item \texttt{T2[512]}: Bank 4 (SRC=0x00--0xFF), Bank 5 (SRC=0x00--0xFF)
  \item \texttt{T3[512]}: Bank 6 (SRC=0x00--0xFF), Bank 7 (SRC=0x00--0xFF)
\end{itemize}

%==============================================================
\section{Source Program for Pi Computation}
The source code for the $\pi$ computation program is written in assembler for the ADS4004 tool. It is located at:


\[
\texttt{SOFTWARE/ADS4004/pi4004.src}
\]


The program can be assembled and simulated on a PC. Use memory dumps to verify behavior during execution.

%==============================================================
\section{Printing the Result with 141-PF Printer}
To execute the $\pi$ computation program, set SW8 to ON on the DE10-Lite board and reset the RISC-V 141-PF calculator program. The total computation time at a clock speed of 750~KHz (4004 CPU) is approximately 17 minutes.

The final result—500 digits of $\pi$—is printed by the 141-PF printer. The following is an example display from the serial terminal:

\begin{verbatim}
===== pi4004 ===== [CUI]
Loading...Done
     1415926535      
     8979323846      
     2643383279      
     5028841971      
     6939937510      
     5820974944      
     5923078164      
     0628620899      
     ..........
     0921861173      
     8193261179      
     3105118548      
     0744623799      
     6274956735      
     1885752724      
     8912279381      
     8301194912
\end{verbatim}


%==============================================================




